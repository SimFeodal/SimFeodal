\documentclass[a4paper,11pt]{article}
\usepackage[T1]{fontenc}
\usepackage[utf8x]{inputenc}
\usepackage{lmodern}
%\usepackage[français]{babel}
\usepackage{amsmath}
\usepackage{graphicx}
\usepackage[hidelinks]{hyperref}
\usepackage{listings}


%%%%%%%%%%%%%%%%
% Commentaires %
%%%%%%%%%%%%%%%%
\setlength{\marginparwidth}{3.5cm} % Pour des commentaires plus grands
\usepackage[colorinlistoftodos,prependcaption,textsize=tiny]{todonotes}
\usepackage{soul}
\newcommand{\hlc}[2][yellow]{ {\sethlcolor{#1} \hl{#2}} }
\definecolor{fluorescentorange}{rgb}{1.0, 0.75, 0.0}
\definecolor{lightred}{rgb}{1, 0.4, 0.4}


\makeatletter
\if@todonotes@disabled
\newcommand{\todohl}[2]{#1}
\else
\newcommand{\todohl}[2]{\texthl{#1}\todo{#2}}
\fi
\makeatother
%%%%%%%%%%%%%%%%
%%%%%%%%%%%%%%%%


\title{\textbf{SimFeodal}\\Fixation, polarisation et hiérarchisation de l'habitat rural en Europe du Nord-Ouest entre 800 et 1100}
\author{Robin Cura \\\textit{UMR Géographie-cités - Université Paris 1} \\ Cécile Tannier \\ \textit{UMR Chrono-Environnement - CNRS et Université Bourgogne Franche-Comté}}
\date{Documentation : Version 11 (2017-10-27)\linebreak
	Modèle : Version 4\_5A\footnote{\hlc[fluorescentorange]{Il s'agit bien de cette version (et du commit associé), encore assez peu testée, puisqu'on décrit les règles de mobilité des FP non mobiles (serfs) qui ne se déplaçaient pas du tout jusqu'à cette version.}} (Commit : \href{https://github.com/RCura/SimFeodal/commit/66e397f3fad13c27685f2740f5c88266ca50fe44}{66e397f})
}

\begin{document}

\maketitle
\tableofcontents

\begin{abstract}
\end{abstract}

\section{Notes introductives}
Ce modèle est développé sur la plate-forme GAMA, dans sa version de développement (future v1.7).

\section{Situation initiale}
Une simulation démarre en l'an 800 (paramètre \textit{debut\_simulation}) et se termine en 1160 (paramètre \textit{fin\_simulation}). Un pas de temps de simulation (i.e. une itération du modèle) représente une durée de 20 ans par défaut (paramètre \textit{duree\_step}), ce qui correspond approximativement à la durée de vie d'une génération à l'époque médiévale. Le nombre de pas de temps d'une simulation est donc égal à : (\textit{debut\_simulation} - \textit{debut\_simulation}) / \textit{duree\_step}).

\subsection{L'environnement}
L'espace modélisé est une zone carrée de $100 \times 100$ kilomètres représentant de manière simplifiée une région de l'Europe du Nord-Ouest en 800.

\subsection{Les foyers paysans}

À l'initialisation, il y a par défaut 4000 foyers paysans (paramètre \textit{nombre\_foyers\_paysans}), répartis dans l'espace du modèle. A leur création, ils ont une probabilité égale à 0.8 par défaut d'être mobiles (paramètre \textit{taux\_mobilite}).

\paragraph{Création des foyers paysans dans des petites villes}
À l'initialisation, par défaut, 3 petites villes pré-existent (paramètre \textit{nombre\_agglos\_antiques}). Elles correspondent à des agglomérations secondaires antiques. Ces petites villes consistent, dans le modèle, en des agrégats de 30 foyers paysans dont chacun est espacé au plus de 100 m. par défaut (paramètre \textit{distance\_detection\_agregats}) de son plus proche voisin. Le fait qu'une agglomération secondaire antique comporte aussi des artisans (et sans doute des prêtres, magistrats...) n'est pas modélisé.

La création des petites villes se fait ainsi :
\begin{itemize}
  \item On crée \textit{nombre\_agglos\_antiques} foyers paysans dans l'espace du modèle raccourci de $1km$ de chaque côté.
  \item A proximité de chacun de ces foyers paysans, on crée 29 (soit \textit{nombre\_agglos\_antiques} - 1) autres foyers paysans situés à moins de \textit{distance\_detection\_agregats} m. les uns des autres. Le nombre total de foyers paysans localisés dans une agglomération antique est donc de \textit{nombre\_agglos\_antiques} $\times$ 30.
\end{itemize}

\paragraph{Création des foyers paysans dans des villages}
À l'initialisation, par défaut, 20 villages pré-existent (paramètre \textit{nombre\_villages}). Un village est défini comme un petit agrégat de foyers paysans, c'est-à-dire comportant par défaut 10 foyers paysans (paramètre  \textit{nombre\_FP\_village}) espacés au plus de \textit{distance\_detection\_agregats} m. les uns des autres.

Règle de création : identique à celle de création des petites villes.

\paragraph{Création des foyers paysans dispersés}
On répartit ensuite le reste des foyers paysans (\textit{nombre\_foyers\_paysans} $-$ ((\textit{nombre\_foyers\_villages} $\times$ \textit{nombre\_villages}) $+$ (30 $\times$ \textit{nombre\_agglos\_antiques})) de manière aléatoire dans l'emprise spatiale du modèle.

\subsection{Les églises}
Initialement, dans l'espace du modèle de $100 \times 100$ km, on place aléatoirement, par défaut, $150$ églises (paramètre \textit{nombre\_eglises}). Parmi elles, par défaut, 50 d'entre elles tirées aléatoirement ont des droits paroissiaux (paramètre \textit{nb\_eglises\_paroissiales}).

\subsection{Les seigneurs}

On distingue trois types de seigneurs :
\begin{itemize}
\item Les \textit{Grands Seigneurs} : il y en a \textit{nombre\_grands\_seigneurs} (2 par défaut). Ils n'ont pas d'existence spatiale propre dans le modèle, c'est-à-dire qu'ils sont extérieurs à l'espace modélisé dans lequel ils interviennent. À  l'initialisation, ils détiennent chacun de nombreuses terres et collectent un loyer pour la terre auprès des foyers paysans qui leur sont assujettis.
On définit ceux-ci comme étant ceux qui ne paient pas déjà un loyer à un autre seigneur.
Chaque grand seigneur se verra attribuer une part de ces paysans correspondante au ratio de sa puissance sur l'ensemble des puissances. Leur puissance relative (\textit{puissance\_grand\_seigneur1 et 2}) vaut 0.5 chacun. Chacun prélèvera donc un loyer à 50\% des FP « libres ».

\item Les \textit{Seigneurs Châtelain} : au départ, il n'y en a pas.

\item Les \textit{Petits Seigneurs} : à l'initialisation, il y en a 18 par défaut (paramètre \textit{nombre\_petits\_seigneurs}). Ils sont localisés dans un agrégat de foyers paysans. Ils collectent un loyer pour la terre auprès de chaque foyer paysan qui leur est assujetti. Les foyers paysans qui leur sont assujettis sont localisés dans leur voisinage. Ce voisinage est de taille variable (un paramètre du modèle sous la forme d'un intervalle de distance ; tirage aléatoire de la taille du voisinage de chaque petit seigneur à l'initialisation du modèle. La valeur de ce paramètre à l'initialisation peut être différente de la valeur en cours de simulation). Les foyers paysans localisés dans le voisinage d'un petit seigneur ne lui paient pas tous un loyer. Certains d'entre eux paient un loyer à l'un des grands seigneurs.
\end{itemize}

\subsection{Châteaux}
L'espace modélisé au départ de la simulation ne comporte aucun château. Les grandes enceintes collectives (castra) qui existent à l'époque ne sont pas représentées dans le modèle.


\section[Attributs et règles]{Attributs et règles de comportement des agents}

\subsection{Les agrégats de foyers paysans}

Les agrégats de foyers paysans sont identifiés (détectés) à la fin de chaque pas de simulation. L'évolution de chacun en cours de simulation n'est pas enregistrée (absence de suivi de chaque agrégat d'un pas de simulation à un autre).

\paragraph{Détection}
Un agrégat de foyers paysans est défini comme un ensemble d'au moins \textit{nombre\_FP\_agregat} foyers paysans. On détecte ces agrégats au moyen d'une agrégation de proche en proche :
\begin{itemize}
  \item (1) un agrégat (\textit{agregats\_detectes}) rassemble les foyers paysans et les attracteurs (églises paroissiales, châteaux et communautés) espacés les uns des autres de \textit{distance\_detection\_agregats} (100 m. par défaut) au maximum.
  \item (2) On crée l'enveloppe convexe de chaque \textit{agregats\_detectes} que l'on élargit de \textit{distance\_fusion\_agregat} (100 m. par défaut).
  \item (3) Puis on fusionne les agrégats détectés élargis qui s'intersectent. On obtient ainsi la liste finale des agrégats.
\end{itemize}

\paragraph{Représentation}
Chaque agrégat est représenté sous forme polygonale, laquelle est constituée par l'enveloppe convexe élargie de 100 m. des foyers paysans la composant.

\paragraph{Attributs}
Les agrégats ont des attributs : présence ou non d'une communauté, nombre d'églises paroissiales, présence d'un château (dans ou à moins de 200 m. de l'agrégat), nombre de foyers paysans. Ils héritent de la communauté d'un agrégat existant au pas de simulation précédent, dont l'enveloppe convexe les intersecte. Quand un agrégat intersecte celle de plusieurs agrégats pré-existants, l'agrégat hérite des attributs de l'un des agrégats pré-existants tiré au hasard.

\paragraph{Apparition de communautés villageoises}
A chaque pas de temps, chaque agrégat a une certaine probabilité

(paramètre $proba\_apparition\_communaute$), égale à 0.2, de voir émerger une communauté villageoise en son sein. Une fois qu'un agrégat possède une communauté, il ne peut la perdre.

Une communauté peut être plus ou moins puissante (paramètre {\textit{puissance\_communaute}). Cette puissance peut varier d'une simulation à une autre mais est identique pour toutes les communautés d'une simulation donnée. L'ensemble des foyers paysans d'un agrégat ayant une communauté font partie de ladite communauté.

\subsection{Seigneurs}

\subsubsection{Apparition de nouveaux petits seigneurs}

Des petits seigneurs apparaissent petit à petit au cours de la simulation : on définit le nombre de seigneurs que l'on souhaite obtenir en fin de simulation, (\textit{nombre\_seigneurs\_objectif}), on soustrait à ce nombre les seigneurs (grands et petits) créés à l'initialisation de la simulation, et on divise le nombre de seigneurs à créer (\textit{nb\_seigneurs\_a\_creer\_total}) par le nombre de pas de temps afin d'obtenir un nombre moyen (entier) de petits seigneurs à  créer à chaque pas de temps (\textit{nb\_moyen\_petits\_seigneurs\_par\_tour}).
A chaque pas de temps, le nombre de petits seigneurs créés est un nombre aléatoire dont la moyenne est \textit{nb\_moyen\_petits\_seigneurs\_par\_tour} et qui s'écarte au maximum d'un tiers de cette valeur. Ces seigneurs sont localisés dans un agrégat choisi aléatoirement.
A leur création, ils n'ont aucun droit, sauf éventuellement le droit de prélever un loyer, après tirage selon la probabilité (\textit{proba\_collecter\_loyer}). En cas de succès, ils créent alors une zone de prélèvement loyer autour d'eux, dont le diamètre est aléatoirement tiré entre \textit{rayon\_min\_PS} et \textit{rayon\_max\_PS} et dont le taux de prélèvement vaut 100\%.

Petits seigneurs : le taux de prélèvement d'une zone de prélèvement de loyer vaut 100\%.
Grands seigneurs : 

\subsubsection{Création de châteaux}

A partir de 960 (paramètre \textit{apparition\_chateaux}) et au fur et à mesure de la simulation, des châteaux sont créés. Chaque château nouvellement créé doit être éloigné d'au moins $3000$ m (si créé par un petit seigneur) ou $5000$ m (si créé par grand seigneur) des châteaux existants.

\bigskip
Un grand seigneur peut créer un ou plusieurs châteaux, soit au sein d'un agrégat qui n'en contient pas encore (avec une probabilité égale à 0.5 - paramètre \textit{proba\_chateau\_agregat}), soit en dehors des agrégats existants. La probabilité pour un grand seigneur de créer un ou des châteaux croît au fur et à mesure de l'augmentation de sa puissance : 

\begin{equation}
proba\_creer\_chateau\_GS = 1 - e^{-n \times puissance\_seign}
\end{equation}
avec $n \approx 6 \times 10^{-4}$.

\bigskip
Un petit seigneur ou un seigneur châtelain suffisamment puissant peut aussi créer un château. A ce titre, il prélève les loyers et les droits banaux et de basse et moyenne justice associés au château. Ce château est forcément localisé à proximité du seigneur, c'est-à-dire dans l'agrégat de population le plus proche de lui, si celui-ci ne possède pas déjà un château. La probabilité, pour un petit seigneur ou un seigneur châtelain, de créer un château varie linéairement entre $0$ si \textit{puissance\_seign} $= 0$
et $1$ si \textit{puissance\_seign} $≥ 2000$.

\bigskip
Entre 940 et 1040 (inclus), certains châteaux peuvent devenir de gros châteaux. La probabilité qu'un château isolé (hors pôle) devienne un gros château (paramètre \textit{proba\_promotion\_groschateau\_autre}) est égale à 0.3. La probabilité qu'un château situé à proximité d'un pôle devienne un gros château (paramètre \textit{proba\_promotion\_groschateau\_multipole}) est égale à 0.8.

\subsubsection{Prélèvement des redevances et droits seigneuriaux}

Le prélèvement des redevances et droits seigneuriaux se fait au sein de zones de prélèvement, dont les attributs sont le détenteur (seigneur qui l'a créée), le rayon et le taux de prélèvement.

\bigskip
\textbf{Prélèvement des loyers}

La totalité des foyers paysans payent un loyer. Ceux situés dans une zone de prélèvement de loyer s'en acquittent au seigneur possédant la zone. Quand les foyers paysans ne sont pas situés dans une telle zone ou que celle-ci n'a pas un taux de prélèvement de 100\%, ce sont les grands seigneurs qui prélèvent ce loyer.

Les petits seigneurs créés à l'initialisation du modèle possèdent chacun une zone de prélèvement de loyers, d'un rayon \textbf{préciser} et d'un taux de prélèvement égal à 100\%.

Les petits seigneurs apparaissant en cours de simulation peuvent créer, selon une probabilité de 0.1 (paramètre \textit{proba\_collecter\_loyer}), une zone de prélèvement de loyers d'un taux de prélèvement compris entre :

5\% (paramètre \textit{min\_fourchette\_loyers\_PS})

et 25\% (paramètre \textit{max\_fourchette\_loyers\_PS}).

% Proposition RC - Enlever la balise \hl{} après validation
\hl{Ainsi, au fur et à mesure de l'avancement de la simulation, les grands seigneurs perdent une part des loyers qu'ils collectaient, c'est-à-dire ceux des foyers paysans hors zone de prélèvement loyer qui sont donc de plus en plus nombreux à payer un loyer au sein d'une zone de prélèvement puisque celles-ci recouvrent de plus en plus le territoire modélisé.
Cependant, comme les plus grandes zones de prélèvement de loyers qui émergent en cours de simulation sont liées à des châteaux (cf. ci-après) et que la plupart des châteaux sont détenus par les grands seigneurs, ceux-ci ne perdent pas ou de très peu de revenus.}

%%%%%%%%%%
% Supprimer ces questions/réponses après lecture. %
%\textbf{de Cécile à Robin : à valider ou à modifier} : Ainsi, chaque grand seigneur perd le prélèvement du loyer hors zone de prélèvement sur chaque foyer paysan qui devient assujetti à une zone de prélèvement de loyers nouvellement créée. Cependant, comme les plus grandes zones de prélèvement de loyers qui émergent en cours de simulation sont liées à des châteaux (cf. ci-après) et que la plupart des châteaux sont détenus par les grands seigneurs, ceux-ci ne perdent pas ou de très peu de revenus.
%%%%%%%%%%

Additionnellement, on remarque qu'au départ d'une simulation, le prélèvement d'un loyer par les grands seigneurs auprès des foyers paysans se fait hors zone de prélèvement. Au cours de la simulation, au fur et à mesure de la création de châteaux, le prélèvement de loyer est de plus en plus associé aux châteaux.

\bigskip
\textbf{Prélèvement des droits de banaux et de droits de basse et moyenne justice}

Au cours de la simulation, les petits seigneurs acquièrent des droits banaux et/ou de basse et moyenne justice sur les foyers paysans qui leur sont assujettis. On le modélise par la probabilité pour un petit seigneur, à chaque pas de simulation, de créer une zone de prélèvement de taille modeste dans son voisinage. Cette probabilité est égale à $0.05$ pour chacun des types de droits :\\
Paramètres :
\begin{itemize}
\item \textit{proba\_creation\_ZP\_banaux} et
\item \textit{proba\_creation\_ZP\_basseMoyenneJustice}.
\end{itemize}

%%%%%%%%%%
% Supprimer ces questions/réponses après lecture. %
\hlc[fluorescentorange]{Questions CT : est-ce que les grands seigneurs prélèvent ces droits hors ZP liées à un château et hors ZP créée par cession de droits ?\\
RC : Non, les GS ne construisent aucune ZP hors des châteaux en dehors de celles crées pour don aux vassaux. Mais ils prélèveront des droits (de suzerain) sur ces ZP banaux/b-m justice qu'ils créent pour don aux PS.\\
CT : Est-ce que les petits seigneurs créés à l'initialisation du modèle ont des ZP pour ces droits ?\\
RC : A l'initialisation des PS, on leur crée une ZP de loyer uniquement. Par la suite, tous les PS (initiaux y compris) ont une probabilité de créer des ZP banaux ou b-m justice.
La seule différence (maintenant, il y en avait plus dans les versions précédentes du modèle) entre PS initial et PS créé en cours est donc la création systématique d'une ZP loyer pour les PS initiaux alors qu'elle est probabiliste (10\%) pour les PS créés pendant la simulation.
}
%%%%%%%%%%

\bigskip
\textbf{Prélèvement des droits de haute justice}

\begin{itemize}
	\item Grands Seigneurs : Un grand seigneur peut acquérir des droits de haute justice à chaque pas de simulation selon une probabilité égale à $0.1$ avant 1000 et égale à $1$ à partir de 1000.
	Une fois qu'il a obtenu ces droits, il crée autour de chacun de ses châteaux (et créera autour de chaque nouveau château qu'il construira) une zone de prélèvement de ces droits. S'il n'a pas de château, il ne prélève donc aucun droit de haute justice. --> Création banaux + BMjustice
	\begin{sloppypar} % Meilleure gestion des césures
	\item Petits Seigneurs châtelains : Lorsqu'il construit un château après 1000, un petit seigneur a une probabilité  \textit{proba\_gain\_droits\_hauteJustice\_chateau} égale à $0.1$ de créer une zone de prélèvement de droits de haute justice autour de ce château. Ce tirage n'est pas rétro-actif : un château créé sans droits de haute justice ne les acquerra jamais. --> Création banaux + BMjustice
	\end{sloppypar}
	\item Petits Seigneurs vassaux : Les petits seigneurs peuvent aussi acquérir une zone de prélèvement de haute justice par le don d'un grand seigneur. Cf. \ref{cession-droits} p.\pageref{cession-droits}.
\end{itemize}

Dans l'ensemble de ces cas, tous les foyers paysans situés dans une zone de prélèvement des droits de haute justice s'acquittent de ces droits au seigneur détenteur de la zone.
La création d'une zone de prélèvement de droits de haute justice autour d'un château (hors don aux petits seigneurs vassaux) entraîne nécessairement la création de deux autres zones de prélèvement de même rayon et de même détenteur correspondant au prélèvement de droits banaux et de droits de basse et moyenne justice.


%%%%%%%%%%
% Supprimer ces questions/réponses après lecture.
\hlc[fluorescentorange]{CT : Comment se font les prélèvements si le seigneur n'a pas encore construit de château et donc ne possède pas encore de zones de prélèvement ?\\
RC : Il n'y a pas de ZP haute justice en dehors des châteaux. Donc, si pas ded chateaux, pas de collecte de droits de haute justice. J'ai réorganisé et completé le descriptif pour que ce soit plus clair.}\\
%%%%%%%%%%




\bigskip
\textbf{Zones de prélèvement liées à un château}

La création d'un château implique nécessairement celle d'une zone de prélèvement de loyers associée dont le rayon dépend de la puissance du propriétaire du château comparée au minimum et maximum de puissance des autres seigneurs au moment de la création de la zone :
\begin{equation}
\begin{gathered}
rayon\_chateau =\\
\frac{max(puissance)}{max(puissance) - min(puissance)} - \frac{puissance_{seigneur}}{max(puissance)- min(puissance)}
\end{gathered}
\end{equation}
avec $minRayon \leq rayon\_chateau \leq maxRayon$ et $minRayon = 2 000$ m et $maxRayon = 10 000$ m.

Tous les foyers paysans situés à l'intérieur de cette zone de prélèvement s'acquittent d'un loyer au seigneur détenteur de la zone.

\bigskip
La création d'un château implique également souvent la création de zones de prélèvement d'autres droits de même rayon que la zone de prélèvement de loyers.
\begin{sloppypar}
\hl{Ainsi, au cours du temps, les petits seigneurs acquièrent (selon une probabilité \textit{proba\_gain\_droits\_banaux\_chateau} et \textit{proba\_gain\_droits\_basseMoyenneJustice\_chateau} égale à $0.1$) le droit de créer des zones de prélèvement de droits banaux ou de droits de basse et moyenne justice. Une fois que ce droit est acquis, chacun de leurs châteaux, anciennement créés comme potentiellement créé par la suite) servira de support à la création d'une zone de prélèvement de type correspondant au droit acquis.}
\end{sloppypar}

\medskip
% Texte initial pour rappel : à supprimer après validation du paragraphe précédent.
%Pour les petits seigneurs, la probabilité pour qu'une zone de prélèvement de droits banaux ou de basse et moyenne justice soit associée à la création d'un château est égale à $0.1$ (paramètres \textit{proba\_gain\_droits\_banaux\_chateau} et \textit{proba\_gain\_droits\_basseMoyenneJustice\_chateau}).\\

%%%%%%%%%%
% Supprimer après lecture %
\hlc[fluorescentorange]{CT : Une fois un château créé, aux pas de simulation suivants, un petit seigneur a-t-il la même probabilité de créer une zone de prélèvement de droits banaux ou de basse et moyenne justice autour de son château ? Comment cela se passe-t-il quand le petit seigneur est détenteur de plusieurs châteaux ?\\
RC : A chaque pas de temps, les PS mettent à jour leurs droits. S'ils ont obtenus les droits banaux ou  les droits b-m justice (selon les proba. du paragraphe au dessus), ils créeront alors autour de tous leurs châteaux une ZP correspondante, de 10km et avec un taux de 100\%. Cela s'applique à tous les châteaux, et seulement pour ceux dont ces ZP n'ont pas été créées.
Donc si un PS crée un château au pas 1 (pas possible mais pas grave) et n'a pas les droits banaux, il n'aura pas de ZP banaux. Si il gagne ces droits au pas 2, il crée alors cette ZP banaux. S'il construit un château au pas 3, il créera alors une ZP banaux (puisqu'il en a gagné le droit au pas 2) autour de ce château au pas 4 (puisque la construction de château se fait en fin de pas, il y a donc un décallage d'un pas de temps). Chaque château peut avoir une seule ZP de chaque type.
J'ai repris la phrase au dessus (en jaune) pour que ce soit plus clair.
}\\
%%%%%%%%%%

Si le petit seigneur constructeur d'un ou plusieurs châteaux acquière des droits de haute justice, il crée alors aussi automatiquement des zones de prélèvement de droits banaux et de droits de basse et moyenne justice autour de chacun de ses châteaux.

\begin{sloppypar} % Meilleure gestion des césures
Les grands seigneurs créent nécessairement autour de tous leurs châteaux les zones de prélèvement des droits qu'ils détiennent. Ainsi, s'ils détiennent des droits banaux, tous leurs châteaux en seront pourvus, et de même pour les droits de basse et moyenne justice. S'ils détiennent des droits de haute justice, tous leurs châteaux seront pourvus des zones de prélèvement correspondantes. Une fois un château créé, entre 900 et 980, les grands seigneurs qui ne détiennent pas encore de droits de haute justice peuvent créer des zones de prélèvement de droits banaux ou de basse et moyenne justice selon une probabilité de $0.1$ à chaque pas de simulation (paramètres \textit{proba\_gain\_droits\_banaux\_chateau} et paramètre \textit{proba\_gain\_droits\_basseMoyenneJustice\_chateau}.\\
\end{sloppypar}	

Finalement, à partir de 1000, comme les grands seigneurs ont forcément tous acquis des droits de haute justice, ils prélèvent chacun les loyers et les différents types de droits (basse, moyenne et haute justice, et droits banaux) autour de tous leurs châteaux.


\subsubsection{Cession de droits}\label{cession-droits}

A partir de 900, les seigneurs cèdent une partie de leurs droits à des petits seigneurs situés dans un voisinage de 3000m. autour d'eux, en échange de leur fidélité.

\bigskip
Un grand seigneur peut céder des droits à un petit seigneur qui n'est pas déjà le vassal d'un autre grand seigneur : dans ce cas, il crée une nouvelle zone de prélèvement. Celle-ci est localisée dans un agrégat, pris au hasard parmi tous. Elle a 1/3 de chance d'être de droits banaux, de droits de haute justice ou de droits de basse et moyenne justice, et son rayon est défini aléatoirement dans un intervalle donné (même principe que pour les petits seigneurs s'arrogeant de nouveaux droits). Le taux de prélèvement (i. e. la part de foyers paysans prélevés dans la zone) est fixé à 100\%. Le petit seigneur récipiendaire du don devient vassal du grand seigneur.

\bigskip
Les petits seigneurs qui détiennent des zones de prélèvement peuvent aussi céder tout ou partie des droits qu'ils possèdent dessus : pour chaque zone de prélèvement qu'ils détiennent, ils ont une probabilité (33\% à chaque pas de simulation) d'en céder une partie, par pas de 5\%, à un autre petit seigneur éloigné de moins de 3 km. L'attribution des redevances des foyers paysans aux petits seigneurs se fait de la même manière que pour les grands seigneurs Ainsi, chaque zone de prélèvement a une liste de récipiendaires associés à un taux de perception des redevances de la zone. La somme de ces taux vaut 100\% quand toute la zone de prélèvement a été cédée par le seigneur détenteur à un ou plusieurs seigneurs récipiendaires.

\subsubsection{Dons de châteaux en garde}

A partir de 950, à chaque pas de simulation, un grand seigneur peut donner certains de ses châteaux en garde à un seigneur de moindre lignage, dont le suzerain n'est pas un autre grand seigneur. Pour chaque château possédé, la probabilité d'être donné en garde est égale à $0.5$ (paramètre \textit{proba\_don\_chateau\_GS}). Le seigneur gardien de château devient alors vassal du grand seigneur et seigneur châtelain (s'il ne l'est pas déjà). Il devient aussi récipiendaire des zones de prélèvement associées au château, y compris des zones de prélèvement des droits de haute justice :
\begin{itemize}
\item Sur la zone de prélèvement\_loyer, il gagne 100\% des droits.
\item Sur les 3 autres types de zones de prélèvement, il gagne 33\% des droits lors du don du château puis 33\% des droits aux deux pas de simulation suivants (99\% est ramené à 100\%).
\end{itemize}


\subsubsection{Montant des redevances perçues}
Pour chaque foyer paysan assujetti, le seigneur perçoit :
\begin{itemize}
	\item Haute justice : Suzerain : $1.25$ \& Vassal : $1$
	\item Basse et moyenne justice : Suzerain : $0.35$ \& Vassal : $0.25$
	\item Droits banaux : Suzerain : $0.35$ \& Vassal : $0.25$
	\item Loyer : $1$\footnote{\hlc[lightred]{Il y a un comportement anormal ici ! Les loyer sont collectés uniquement par les GS (FP hors ZP loyer) et par les propriétaires (== créateurs) d'une ZP loyer (PS et GS, ZP châteaux et autres). Or, les ZP loyers peuvent être données en gardiennage dans le cadre (1) des dons de châteaux par les GS à des vassaux et (2) des dons de ZP (hors châteaux) des PS à d'autres PS proches. Ces dons n'ont donc strictement aucun effet dans le cadre de ces ZP loyers. Ça pourrait poser problème en augmentant artificiellement le nombre de vassaux des seigneurs (ce qui n'a pas d'effet dans le modèle mais qu'on a voulu observer de temps en temps) alors que ceux-ci n'en tirent aucun avantage concret.}}
\end{itemize}
\subsubsection{Calcul de la puissance des seigneurs}

La puissance d'un seigneur est calculée par le modèle à chaque pas de simulation. Elle se présente sous deux formes. 
\begin{itemize}
\item Puissance issue des redevances perçues (paramètre \textit{puissance\_seign}) : somme des redevances perçues.
\item Puissance armée d'un seigneur : nombre total de foyers paysans qui lui sont assujettis (quels que soient les droits concernés et que ce soit via une zone de prélèvement dont il est détenteur ou récipiendaire).
\end{itemize}


\subsection{Églises paroissiales}

Les polygones de Thiessen permettent de découper l'espace du modèle de telle sorte que chaque église paroissiale soit au centre d'une zone définie comme étant son aire de desserte (i.e. son ressort paroissial en fin de période de simulation). 

A chaque pas de simulation, de nouvelles églises paroissiales apparaissent dans ou à proximité d'agrégats de foyers paysans, selon une probabilité égale à :
\begin{equation}
\begin{gathered}
proba\_creation\_paroisse =\\
( \frac{1}{seuil\_creation\_paroisse}) \times ( \frac{nb\_FP\_agregat}{nb\_paroisses\_agregat} )
\end{gathered}
\end{equation}

Le terme \textit{nb\_FP\_agregat} correspond au nombre de foyers paysans dans l'agrégat et le terme \textit{nb\_paroisses\_agregat} au nombre de ressorts paroissiaux inclus dans, ou intersectant, l'agrégat entouré d'une zone tampon de 200 m de large. La valeur du paramètre \textit{seuil\_creation\_paroisse} est fixée par défaut à 300.

Une fois ces nouvelles églises paroissiales créées, l'espace du modèle fait l'objet d'un nouveau découpage afin d'actualiser les contours des ressorts paroissiaux.

Ensuite, au sein de chaque ressort paroissial, on calcule la satisfaction religieuse de chaque foyer paysan. Si \textit{nb\_paroissiens\_mecontents} (nombre de foyers paysans de la paroisse ayant une satisfaction religieuse égale à 0) > \textit{nb\_paroissiens\_mecontents\_necessaires} (nombre requis de foyers paysans mécontents, d'une valeur par défaut égale à 20), une nouvelle paroisse est créée selon les règles suivantes, dépendant du nombre d'églises non paroissiales dans le ressort :
	\begin{itemize}
\item supérieur à 3, la plus éloignée d'entre elles de l'église paroissiale existante devient paroissiale (application d'une triangulation de Delaunay aux églises non paroissiales de l'espace du modèle ; sélection des trois églises appartenant au triangle le plus proche de l'église paroissiale existante ; parmi celles-ci, sélection de celle qui est la plus éloignée de l'église paroissiale existante) ;
\item entre 1 et 3, l'une d'entre elles (au hasard) devient paroissiale ;
\item égal à 0, s'il en existe à moins de 2 km alentours, l'une d'entre elles (au hasard) devient paroissiale ;
\item dans les autres cas, une église paroissiale est créée dans le ressort paroissial, au sommet du polygone de Thiessen le plus éloigné de l'église paroissiale existante.
	\end{itemize}
	
L'espace du modèle fait alors l'objet d'un nouveau découpage afin d'actualiser les contours des ressorts paroissiaux et de tenir compte des nouvelles paroisses lors du calcul de la satisfaction religieuse des foyers paysans.


\subsection{Pôles d'attraction}

Un pôle d'attraction est constitué d'un ou plusieurs attracteurs proches les uns des autres de 200 m. On crée l'enveloppe convexe de chaque pôle que l'onélargit de 200 m. Dans le cas où un ou plusieurs pôles intersectent un agrégat, on effectue l'union géométrique de l'enveloppe du ou des pôles (fusionnés alors en un seul pôle) et de l'agrégat.

\bigskip
Les attracteurs considérés dans le modèle sont les châteaux, les églises paroissiales et les communautés (représentées par le centroïde de l'agrégat dont elles relèvent).
\begin{itemize}
	\item Valeur d'attraction d'un petit château : $0.15$
	\item Valeur d'attraction d'un gros château : $0.25$.
	\item Valeur d'attraction d'une église : $0.15$
	\item Valeur d'attraction de deux églises : $0.25$
	\item Valeur d'attraction de trois églises : $0.50$
	\item Valeur d'attraction de quatre églises et plus : $0.60$
	\item Valeur d'attraction d'une communauté : $0.15$
\end{itemize}

La valeur d'attraction d'un pôle sur un foyer paysan est comprise entre $0$ et $1$ et correspond à la somme des valeurs d'attraction des attracteurs du pôle.

\subsection{Foyers Paysans}

\subsubsection{Disparition et Apparition}
A chaque pas de temps, chaque foyer paysan a une probabilité égale à 0.05 par défaut (paramètre \textit{taux\_renouvellement}) de disparaître.

Le même nombre de foyers paysans est recréé, de sorte que le nombre de foyers paysans soit toujours égal à \textit{nombre\_foyers\_paysans}. Ces foyers paysans nouvellement créés ont la probabilité $taux\_mobilite$ de pouvoir entreprendre un déplacement lointain.

Les foyers paysans apparaissant en cours de simulation sont placés dans un agrégat choisi suivant un tirage aléatoire pondéré par le nombre de foyers paysans présents dans chaque agrégat.

\subsubsection{Satisfaction}
La satisfaction d'un foyer paysan conditionne son éventuel déplacement local ou lointain (i.e. migration). La satisfaction consiste en l'agrégation de trois valeurs de satisfaction intermédiaires (satisfaction matérielle, satisfaction religieuse et satisfaction protection). Absence de compensation entre les trois critères de satisfaction (matérielle, religieuse et de protection).
\begin{equation}
\begin{gathered}
Satisfaction =\\0.75 \times [MIN (S_{mat\acute{e}rielle} ; S_ {religieuse}; S_{protection})] +\\0.25 \times [appartenance\_communaut\acute{e}]
\end{gathered}
\end{equation}

Avec $appartenance\_communaut\acute{e}$ égale à 0 (si non appartenance à une communauté) ou 1 (si appartenance).

\begin{enumerate}
  \item \textbf{Satisfaction matérielle.} 
  C'est une fonction ([0;1]) des redevances dont doit s'acquitter le foyer paysan (plus le foyer paysan doit s'acquitter de redevances, moins il est satisfait) (variable $S_{redevance}$) et de l'appartenance ou non du foyer paysan à une communauté (un foyer paysan est davantage satisfait quand il appartient à une communauté, et d'autant plus satisfait que la communauté à laquelle il appartient est puissante) (paramètre $puissance\_communaute$ égal à 0.25 quand la communauté est peu puissante, égal à 0.75 quand elle est puissante et entre 0.25 et 0.75 quand elle est de puissance intermédiaire).


\begin{equation}\label{eq:Smat}
\begin{gathered}
S_{mat\acute{e}rielle} =\\(S_{redevance})^{(1-puissance\_communaute)}
\end{gathered}
\end{equation}
  
Le montant des redevances seigneuriales dont chaque foyer paysan doit s'acquitter est la somme des droits des seigneurs dont dépend le foyer paysan.

\begin{equation}
\begin{gathered}
S_{redevance} =\\ max [ (1- (redevances\_acquittees / coef\_redevances)) ; 0]
\end{gathered}
\end{equation}

avec le paramètre \textit{coef\_redevances} égal à 15 pour la Touraine et la variable \textit{redevances\_acquittées} $∈ [0,n]$.

\bigskip

  \item \textbf{Satisfaction religieuse.} Elle représente la plus ou moins grande facilité pour le foyer paysan à effectuer la pratique religieuse obligatoire.
  
C'est une fonction ([0;1]) de la distance aux églises paroissiales (satisfaction inversement proportionnelle à la distance à parcourir). Les règles d'évaluation de la distance aux églises paroissiales varient au cours du temps.
\begin{equation}
\begin{gathered}
S_{religieuse} = \\max \left \lbrack \frac{(distance_{max} - distance\_eglise)}{(distance_{max} -distance_{min})}; 1 \right \rbrack
\end{gathered}
\end{equation}

avec
\begin{itemize}
	\item avant 950 : $distance_{min}$ = 5 km et $distance_{max}$ =  25 km
	\item de 950 à 1050 : $distance_{min}$ = 3 km et $distance_{max}$ =  10 km
	\item après 1050 : $distance_{min}$ = 1,5 km et $distance_{max}$ =  5 km
\end{itemize}

\bigskip
  
  \item \textbf{Satisfaction protection.} Elle est fonction de la distance entre le foyer paysan et le château le plus proche ($distance\_chateau$) et du besoin de protection ressenti par le foyer paysan ($besoin\_protection$).

\begin{equation}
S_{protection} = (S_{distance\_chateau})^{(besoin\_protection)}
\end{equation}

avec
\begin{itemize}
	\item avant 960 : $besoin\_protection$ = 0
	\item de 960 à 1020 :$besoin\_protection$ $= 0.2 ; 0.4 ; 0.6 ; 0.8$
	\item $besoin\_protection$ = $1$
\end{itemize}

et

$S_{distance\_chateau} =$ \\
$0 ≤ (dist_{max} - distance_{chateau}) / (dist_{max} - dist_{min}) ≤ 1$

avec $dist_{min}$ = 1,5 km et $dist_{max}$ = 5 km.

\end{enumerate}


\subsubsection{Déplacement}
\begin{sloppypar}
A chaque pas de temps, le foyer paysan à une certaine probabilité de se déplacer localement, dans un rayon inférieur ou égal à \textit{distance\_max\_dem\_local}. Quand le foyer paysan est localisé dans un agrégat, un déplacement local envisagé uniquement si un autre agrégat dans un rayon de \textit{distance\_max\_dem\_local} contient un pôle davantage attractif.
\end{sloppypar}
\begin{equation}
p(deplacement\_local) = 1 - Satisfaction
\end{equation}

La valeur de \textit{distance\_max\_dem\_local} est égale à :
\begin{itemize}
\item 2500 m. entre 800 et 880,
\item 4000 m. entre 900 et 980,
\item 6000 m. à partir de 1000.
\end{itemize}
Si la probabilité de déplacement local ($p(deplacement\_local)$) se réalise, un tirage aléatoire pondéré par les attractivités respectives des pôles d'attraction locaux va déterminer où le foyer paysan va s'implanter. S'il n'existe pas de pôle d'attraction dans un rayon de distance au plus égal à\\ \textit{distance\_max\_dem\_local} m. autour du foyer paysan, celui-ci ne se déplace pas.

\bigskip
Chaque foyer paysan a une probabilité ($taux\_mobilite$, fixée à la création de chaque agent) d'être mobile. Si le foyer paysan n'est pas mobile ($ mobile = false $), il ne peut pas entreprendre de déplacement lointain. C'est le cas des foyers paysans non libres (serfs, esclaves). S'il n'effectue pas de déplacement local, un foyer paysan mobile a une certaine probabilité ($p(deplacement\_lointain)$) que le déplacement effectué soit lointain (i.e. d'une distance supérieure à \textit{distance\_max\_dem\_local}) dans la région modélisée. Les destinations possibles d'un déplacement lointain sont (uniquement) les pôles localisés dans ou à proximité d'un agrégat.
\begin{equation}
p(deplacement\_lointain) = 0,2 \times (1 - Satisfaction)
\end{equation}


\clearpage

\section{Variables / Paramètres : valeurs par défaut}

Un paramètre est une quantité donnée en entrée du modèle, qui reste fixe pour une simulation donnée. Une variable est une quantité dont la valeur change au cours d'une simulation.

\subsection{Inputs}\label{inputs}

\subsubsection{Agrégats}
\begin{itemize}
	
	\item Nombre minimum de foyers paysans nécessaires pour constituer un agrégat :\\
	\textbf{\textit{nombre\_FP\_agregat}} = 5
	
	\item Nombre de foyers paysans contenus dans chaque village à l'initialisation du modèle :\\
	\textbf{\textit{nombre\_FP\_village}} = 10
	
\end{itemize}

\subsubsection{Foyers Paysans}
\begin{itemize}
	
	\item Quantification du besoin de protection des foyers paysans résultant de l'apparition de conflits entre les seigneurs locaux :\\ \textbf{\textit{besoin\_protection}} = 0 entre 800 et 940, = 0.2 en 960, = 0.4 en 980, = 0.6 en 1000, = 0.8 en 1020, = 1 à partir de 1040.
	
	\item Rayons de distance maximale sur laquelle un foyer paysan peut se déplacer lors d'un déplacement local :\\ \textbf{\textit{seuils\_distance\_max\_dem\_local}} = $2500$ avant 900, $4000$ entre 900 et 980, $6000$ à partir de 1000.
	
\end{itemize}

\subsubsection{Seigneurs}
\begin{itemize}
	\item \textbf{\textit{nombre\_seigneurs\_objectif}} = 200 : nombre total de seigneurs (petits et grands) à atteindre en fin de simulation. A chaque pas de temps, on crée $n$ seigneurs avec $n = $\\
	(\textit{nombre\_seigneurs\_objectif} - (\textit{nombre\_grands\_seigneurs} +\\ \textit{nombre\_petits\_seigneurs})) / nombre de pas de temps.
	
	\item \textbf{\textit{nombre\_grands\_seigneurs}} = 2 : nombre de grands seigneurs à l'initialisation (1 ou 2 actuellement)
	
	\item \textbf{\textit{nombre\_petits\_seigneurs}} = 18 : nombre de petits seigneurs à l'initialisation
	
	\item \textbf{\textit{puissance\_grand\_seigneur1}} = 5\\ \textbf{\textit{puissance\_grand\_seigneur2}} = 5\\
	puissance relative des seigneurs, (chaque seigneur a une puissance définie par le ratio entre sa puissance et la somme des puissances), qui affecte le nombre de loyers que chaque grand seigneur percevra
	
	\item \textbf{\textit{proba\_collecter\_loyer}} = 0.1 (10\%) : probabilité, pour un petit seigneur créé pendant la simulation, d'obtenir un droit à collecter des loyers (via la création d'une zone de prélèvement de loyer)
	
	\item \textbf{\textit{proba\_creation\_ZP\_banaux}} = 0.05 (5\%) : probabilité, pour un petit seigneur, de créer une nouvelle zone de prélèvement de droits banaux (même s'il en possède déjà une) dans son voisinage.
	
	\item \textbf{\textit{proba\_creation\_ZP\_basseMoyenneJustice}} = 0.05 (5\%) : Probabilité, pour un petit seigneur, de créer une nouvelle zone de prélèvement de droits de basse et moyenne justice (même s'il en possède déjà une) dans son voisinage.
\end{itemize}

\subsubsection{Zones Prélèvement}
\begin{itemize}
	\item \textbf{\textit{rayon\_min\_PS}} = 1000 (m)\\
	\textbf{\textit{rayon\_max\_PS}}  = 5000 (m)\\
	rayon $min$ et $max$ des zones de prélèvements que les petits seigneurs créeront. Le rayon est tiré aléatoirement, pour chaque nouvelle zone de prélèvement, entre ces deux bornes :
	\begin{equation}\label{eq:fourchette}
	\text{rayon\_zone} = \text{rayon\_min} +
	random\{\text{rayon\_max} - \text{rayon\_min}\}
	\end{equation}

	\item \textbf{\textit{min\_fourchette\_loyers\_PS}} = 0.05 (5\%) \\
	\textbf{\textit{max\_fourchette\_loyers\_PS}} = 0.25 (25\%)\\
	dans les zones de prélèvement qu'ils possèdent, les petits seigneurs prélèvent des droits sur un pourcentage des foyers paysans dont la valeur est comprise entre ces deux bornes. (cf. \autoref{eq:fourchette})
	

	\item \textbf{\textit{proba\_don\_partie\_ZP}} = 0.33 (33\%) : probabilité qu'ont les petits seigneurs de donner une part des zones de prélèvement (une partie de chacune des zones de prélèvement) à d'autres petits seigneurs.
	\begin{itemize}
	\item Si $0$, les petits seigneurs restent propriétaires/exploitants uniques de leurs zones de prélèvement.
	\begin{sloppypar} % Meilleure gestion des césures
	\item Si $1$, les petits seigneurs donneront en exploitation, \textit{in fine}, l'ensemble de leurs zones de prélèvement (par tranches de multiples de 5\%).
	\end{sloppypar}
	\end{itemize}
\end{itemize}

\subsubsection{Châteaux}
\begin{itemize}
	
		\item \textbf{\textit{apparition\_chateaux}} = 960 (ans) : année d'apparition des châteaux.
		
	\item \textbf{\textit{proba\_creer\_chateau\_GS}} = 1.0 (100\%) : probabilité qu'a un grand seigneur de créer un château, s'il possède suffisamment de puissance.
	
	\item \textbf{\textit{proba\_chateau\_agregat}} = 0.5 (50\%) : probabilité pour chaque château de grand seigneur créé, qu'il soit situé dans un agrégat.
	
	\item \textbf{\textit{proba\_don\_chateau\_GS}} = 0.33 (33\%) : à chaque pas de temps, les grand seigneur ont une telle probabilité de donner chacun des châteaux qu'ils possèdent en gardiennage à un petits seigneurs qui n'a pas de suzerain, ou dont il est déjà suzerain.
	
	\item \textbf{\textit{proba\_creer\_chateau\_PS}} = 1.0 (100\%) :\\
	\textit{Idem} \textit{proba\_creer\_chateau\_GS} pour les petits seigneurs.
	
	\item \textbf{\textit{proba\_gain\_droits\_hauteJustice\_chateau}} = 0.1 (10\%) : 
	A partir de l'an 900, les châteaux ont une telle probabilité de créer une zones de prélèvement de droits de haute Justice, s'ils n'en ont pas déjà. Viennent alors directement avec de nouvelles zones de prélèvement de droits banaux et de droits de basse et moyenne justice.
	
	\item \textbf{\textit{proba\_gain\_droits\_banaux\_chateau}} = 0.1 (10\%) : lors de la création d'un nouveau château, ou lorsqu'un seigneur ne possède pas déjà les droits banaux, probabilité de création d'une zone de prélèvement de droits banaux.
	
	\item \textbf{\textit{proba\_gain\_droits\_basseMoyenneJustice\_chateau}}\\= 0.1 (10\%) : \textit{Idem} précédent, pour les droits de basse et moyenne justice.
	
\end{itemize}

\subsubsection{Églises}
\begin{itemize}
	\item \textbf{\textit{nombre\_eglises}} = 150 : nombre d'églises à l'initialisation
	
	\item \textbf{\textit{nb\_eglises\_paroissiales}} = 50 : nombre d'églises paroissiales (et donc de paroisses) à l'initialisation
	
	\item \textbf{\textit{proba\_gain\_droits\_paroissiaux}} = 0.05 (5\%) : probabilité, pour chaque église non paroissiale, d'acquérir des droits paroissiaux, à chaque pas de temps 
	
	\item \textbf{\textit{nb\_min\_paroissiens}} = 10 
	\\ \textbf{\textit{nb\_max\_paroissiens}} = 60\\
	dans les agrégats composés de plus que ce seuil de foyers paysans, une paroisse est créée selon une probabilité dépendante du nombre de paroisses et de foyers paysans déjà présents dans l'agrégat :
	\begin{equation}
	\begin{gathered}
	\text{proba\_creation}=\\max
	\left [ 0, min \left \{  1, \frac{\text{nb\_max\_paroissiens}}{\Delta_{paroissiens} } - \frac{\text{nb\_relatif\_paroissiens}}{\Delta_{paroissiens}}
	\right  \}
	\right ]
	\end{gathered}
	\end{equation}
	avec $\Delta_{paroissiens} = (\text{nb\_max\_paroissiens} - \text{nb\_min\_paroissiens})$
	et $\text{nb\_relatif\_paroissiens}$ le nombre de foyers paysans de l'agrégat divisé par le nombre de paroisses qui l'intersectent.
	
\end{itemize}



%\begin{lstlisting}
%Exemple de  code
%\end{lstlisting}


\end{document}
